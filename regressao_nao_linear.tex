\section{Regressão Não-Linear}

Nestes problemas, nós vamos novamente aplicar Regressão Linear para classificação. Considere a função target

$$f(x_1,x_2) = sign(x_1^2 + x_2^2 - 0.6) $$

Gere um conjunto de treinamento de $N = 1000$ pontos em $\mathcal{X} = [-1, 1] \times [-1, 1]$ com probabilidade uniforme escolhendo cada $x \in \mathcal{X}$. Gere um ruído simulado selecionando aleatoriamente 10\% do conjunto de treinamento e invertendo o rótulo dos pontos selecionados.

\begin{enumerate}
    \item Execute a Regressão Linear sem nenhuma transformação, usando o vetor de atributos $(1, x_1, x_2)$ para encontrar o peso $w$. Qual é o valor aproximado de classificação do erro médio dentro da amostra $E_{in}$ (medido ao longo de 1000 execuções)?
    
    \item gora, transforme os $N = 1000$ dados de treinamento seguindo o vetor de atributos não-linear $(1, x_1, x_2, x_1x_2, x_1^2, x_2^2)$. Encontre o vetor $\tilde{w}$ que corresponda à solução da Regressão Linear. Quais das hipóteses a seguir é a mais próxima à que você encontrou? Avalie o resultado médio obtido após 1000 execuções.
    
    \item Qual o valor mais próximo do erro de classificação fora da amostra $E_{out}$ de sua hipótese na questão anterior? (Estime-o gerando um novo conjunto de 1000 pontos e usando 1000 execuções diferentes, como antes).
\end{enumerate}

