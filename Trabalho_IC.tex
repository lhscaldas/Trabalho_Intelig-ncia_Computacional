\documentclass[12 pt]{article}
\usepackage[utf8]{inputenc}
\usepackage{matlab-prettifier}
\usepackage[portuguese]{babel}
\usepackage{indentfirst}
\usepackage{graphicx}
\usepackage{float}
\usepackage{subcaption}
\usepackage[font=small,labelfont=bf]{caption}
\definecolor{mygreen}{RGB}{28,172,0} % color values Red, Green, Blue
\usepackage{mathtools}
\usepackage{multirow}
\usepackage{comment}
\usepackage{color}
\usepackage{colortbl}
\usepackage[normalem]{ulem}               % to striketrhourhg text
\usepackage{amsmath}
\usepackage{amsfonts}
\newcommand\redout{\bgroup\markoverwith
{\textcolor{red}{\rule[0.5ex]{2pt}{0.8pt}}}\ULon}
\renewcommand{\lstlistingname}{Código}% Listing -> Algorithm
\renewcommand{\lstlistlistingname}{Lista de \lstlistingname s}% List of Listings -> List of Algorithms

\usepackage[top=3cm,left=2cm,bottom=2cm, right=2cm]{geometry}

\lstset{language=Python,
    basicstyle={\color{black}\small\ttfamily},   
    breaklines=true,%
    frame=single,
    backgroundcolor=\color{gray!10},
    %morekeywords={matlab2tikz},
    keywordstyle=\color{blue},%
    morekeywords=[2]{1}, keywordstyle=[2]{\color{black}},
    identifierstyle=\color{black},%
    stringstyle=\color{purple},
    commentstyle=\color{green},%
    showstringspaces=false,%without this there will be a symbol in the places where there is a space
    numbers=left,%
    numberstyle={\tiny \color{black}},% size of the numbers
    numbersep=9pt, % this defines how far the numbers are from the text
    %emph=[1]{for,end,break},emphstyle=[1]\color{red}, %some words to emphasise
    %emph=[2]{word1,word2}, emphstyle=[2]{style}, 
    extendedchars=true,
    literate={á}{{\'a}}1 {à}{{\`a}}1 {ã}{{\~a}}1 {é}{{\'e}}1 {É}{{\'E}}1 {ê}{{\^e}}1 {ë}{{\"e}}1 {í}{{\'i}}1 {ç}{{\c{c}}}1 {Ç}{{\c{C}}}1 {õ}{{\~o}}1 {ó}{{\'o}}1 {ô}{{\^o}}1 {ú}{{\'u}}1 {â}{{\^a}}1 {~}{{$\sim$}}1
}


\title{%
\textbf{Universidade Federal do Rio de Janeiro} \newline \par
\textbf{Instituto Alberto Luiz Coimbra de Pós-Graduação e Pesquisa de Engenharia} \par
\includegraphics[width=8cm]{COPPE UFRJ.png} \par
\textbf{Programa de Engenharia de Sistemas e Computação} \par
CPS844 - Inteligência Computacional I \par
Prof. Dr. Carlos Eduardo Pedreira \par 
\vspace{1\baselineskip}
\textit{Trabalho prático}
}

\author{Luiz Henrique Souza Caldas\\email: lhscaldas@cos.ufrj.br}

\date{\today}

\begin{document}
\maketitle

\newpage

\section{Perceptron}

Neste problema, você criará a sua própria função target $f$ e uma base de dados $D$ para que possa ver como o
Algoritmo de Aprendizagem Perceptron funciona. Escolha $d = 2$ pra que você possa visualizar o problema,
e assuma $\chi = [-1, 1] \times [-1, 1]$ com probabilidade uniforme de escolher cada $x \in \mathcal{X}$ .

Em cada execução, escolha uma reta aleatória no plano como sua função target $f$ (faça isso selecionando dois pontos aleatórios, uniformemente distribuídos em  $\chi = [-1, 1] \times [-1, 1]$, e pegando a reta que passa entre eles), de modo que um lado da reta mapeia pra +1 e o outro pra -1. Escolha os inputs xn da base de dados como um conjunto de pontos aleatórios (uniformemente em $ \mathcal{X}$ ), e avalie a função target em cada $x_n$ para pegar o output correspondente $y_n$.

Agora, pra cada execução, use o Algoritmo de Aprendizagem Perceptron (PLA) para encontrar $g$. Inicie o PLA com um vetor de pesos $w$ zerado (considere que $sign(0) = 0$, de modo que todos os pontos estejam classificados erroneamente ao início), e a cada iteração faça com que o algoritmo escolha um ponto aleatório dentre os classificados erroneamente. Estamos interessados em duas quantidades: o número de iterações que o PLA demora para convergir pra $g$, e a divergência entre $f$ e $g$ que é $\mathbb{P}[f (x) \neq g(x)]$ (a probabilidade de que $f$ e $g$ vão divergir na classificação de um ponto aleatório). Você pode calcular essa probabilidade de maneira exata, ou então aproximá-la ao gerar uma quantidade suficientemente grande de novos pontos para estimá-la (por exemplo, 10.000).

A fim de obter uma estimativa confiável para essas duas quantias, você deverá realizar 1000 execuções do experimento (cada execução do jeito descrito acima), tomando a média destas execuções como seu resultado
final.

Para ilustrar os resultados obtidos nos seus experimentos, acrescente ao seu relatório gráficos scatterplot
com os pontos utilizados para calcular $E_{out}$, assim como as retas correspondentes à função target e à hipótese $g$ encontrada.

\begin{enumerate}
    \item Considere $N = 10$. Quantas iterações demora, em média, para que o PLA convirja com $N = 10$
    pontos de treinamento? Escolha o valor mais próximo do seu resultado.
    
    \item Qual das alternativas seguintes é mais próxima de $\mathbb{P}[f(x) \neq g(x)]$ para $N = 10$?
    
    \item Agora considere $N = 100$. Quantas iterações demora, em média, para que o PLA convirja com
    N = 100 pontos de treinamento? Escolha o valor mais próximo do seu resultado/.

    \item Qual das alternativas seguintes é mais próxima de $\mathbb{P}[f(x) \neq g(x)]$ para $N = 100$?
    
    \item  possível estabelecer alguma regra para a relação entre N , o número de iterações até a convergência,
    e $\mathbb{P}[f(x) \neq g(x)]$?
\end{enumerate}
\section{Regressão Linear}

Nestes problemas, nós vamos explorar como Regressão Linear pode ser usada em tarefas de classificação.
Você usará o mesmo esquema de produção de pontos visto na parte acima do Perceptron, com $d = 2$,
$\mathcal{X} = [-1, 1] \times [-1, 1]$, e assim por diante.

\begin{enumerate}
    \item Considere $N = 100$. Use Regressão Linear para encontrar $g$ e calcule $E_{in}$, a fração de pontos dentro da amostra que foram classificados incorretamente (armazene os $g$'s pois eles serão usados no item seguinte). Repita o experimento 1000 vezes. Qual dos valores abaixo é mais próximo do $E_{in}$ médio?

    \item Agora, gere 1000 pontos novos e use eles para estimar o $Eout$ dos $g$'s que você encontrou no item anterior. Novamente, realize 1000 execuções. Qual dos valores abaixo é mais próximo do $E_{out}$ médio?

    \item Agora, considere $N = 10$. Depois de encontrar os pesos usando Regressão Linear, use-os como um vetor de pesos iniciais para o Algoritmo de Aprendizagem Perceptron (PLA). Execute o PLA até que ele convirja num vetor final de pesos que separa perfeitamente os pontos dentro-de-amostra. Dentre as opções abaixo, qual é mais próxima do número médio de iterações (sobre 1000 execuções) que o PLA demora para convergir?
    
    \item Vamos agora avaliar o desempenho da versão pocket do PLA em um conjunto de dados que não é linearmente separável. Para criar este conjunto, gere uma base de treinamento com N2 pontos como foi feito até agora, mas selecione aleatoriamente 10\% dos pontos e inverta seus rótulos. Em seguida, implemente a versão pocket do PLA, treine-a neste conjunto não-linearmente separável, e avalie seu $E_{out}$ numa nova base de N2 pontos na qual você não aplicará nenhuma inversão de rótulos. Repita para 1000 execuções, e mostre o $E_{in}$ e $E_{out}$ médios para as seguintes configurações (não esqueça dos gráficos scatterplot, como anteriormente):
\end{enumerate}
\section{Regressão Não-Linear}

Nestes problemas, nós vamos novamente aplicar Regressão Linear para classificação. Considere a função target

$$f(x_1,x_2) = sign(x_1^2 + x_2^2 - 0.6) $$

Gere um conjunto de treinamento de $N = 1000$ pontos em $\mathcal{X} = [-1, 1] \times [-1, 1]$ com probabilidade uniforme escolhendo cada $x \in \mathcal{X}$. Gere um ruído simulado selecionando aleatoriamente 10\% do conjunto de treinamento e invertendo o rótulo dos pontos selecionados.

\begin{enumerate}
    \item Execute a Regressão Linear sem nenhuma transformação, usando o vetor de atributos $(1, x_1, x_2)$ para encontrar o peso $w$. Qual é o valor aproximado de classificação do erro médio dentro da amostra $E_{in}$ (medido ao longo de 1000 execuções)?
    
    \item gora, transforme os $N = 1000$ dados de treinamento seguindo o vetor de atributos não-linear $(1, x_1, x_2, x_1x_2, x_1^2, x_2^2)$. Encontre o vetor $\tilde{w}$ que corresponda à solução da Regressão Linear. Quais das hipóteses a seguir é a mais próxima à que você encontrou? Avalie o resultado médio obtido após 1000 execuções.
    
    \item Qual o valor mais próximo do erro de classificação fora da amostra $E_{out}$ de sua hipótese na questão anterior? (Estime-o gerando um novo conjunto de 1000 pontos e usando 1000 execuções diferentes, como antes).
\end{enumerate}





\end{document}